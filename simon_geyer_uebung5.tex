\documentclass{scrartcl}
\usepackage{geometry}
\geometry{verbose,a4paper,tmargin=40mm,bmargin=35mm,lmargin=40mm,rmargin=40mm,nomarginpar}
\setlength{\parindent}{0mm}
\setlength{\parskip}{6pt}

\usepackage[utf8]{inputenc}
\usepackage[ngerman]{babel}
\usepackage{fixltx2e}
\usepackage{kpfonts}
\usepackage[sc]{mathpazo} %[sc,osf] für oldstylenumbers
%\usepackage{eulervm} % <-- falls Formeln falsch angezeigt werden
\usepackage[scaled]{helvet} % ss
\usepackage{courier} % tt
\linespread{1.05} % weil palationo
\normalfont
\usepackage[T1]{fontenc}

\usepackage{microtype}
\usepackage{tabularx}
\usepackage{booktabs}

\author{Simon Geyer}
\date{\today}
\title{CS 102 \LaTeX{} Übung}

\begin{document}
\maketitle

\section{Das ist der erste Abschnitt}
Hier könnte auch anderer Text stehen.

\section{Tabelle}
Unsere wichtigsten Daten finden sie in Tabelle \ref{tab}.
\begin{center}
\begin{tabularx}{0.78\textwidth}{X|c|c|c}
& Punkte erhalten & Punkte möglich & \% \\
\hline
Aufgabe 1 & 2 & 4 & 50\\
Aufgabe 2 & 3 & 3 & 100\\
Aufgabe 3 & 3 & 3 & 100\\
\end{tabularx}
\captionof{table}{Eine Tabelle mit Werten.}
\label{tab}
\end{center}

\section{Formeln}

\subsection{Pythagoras}
Der Satz des Pythagoras errechnet sich wie folgt: $a^2 + b^2 = c^2$ . Daraus können
wir die Länge der Hypothenuse wie folgt berechnen: $c = \sqrt{a^2 + b^2}$.

\subsection{Summen}
Wir können auch die Formel für eine Summe angeben:
\begin{equation}
s = \sum_{i=1}^{n} i = \frac{n\cdot (n + 1)}{2}
\end{equation}
\end{document}
